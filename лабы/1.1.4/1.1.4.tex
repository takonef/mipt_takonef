\documentclass[a4paper]{article}
\usepackage{graphicx} % Required for inserting images
\usepackage{wrapfig}
\usepackage{enumitem}
\usepackage[center]{caption}
\usepackage[english, russian]{babel}
\usepackage{amsmath}
\usepackage{geometry}

\geometry{verbose, a4paper, tmargin=2cm, bmargin=2cm, lmargin=2.0cm, rmargin=2.0cm}

\begin{document}

\begin{titlepage}
	\centering
	{\scshape\Large МОСКОВСКИЙ ФИЗИКО-ТЕХНИЧЕСКИЙ ИНСТИТУТ \\
	(НАЦИОНАЛЬНЫЙ ИССЛЕДОВАТЕЛЬСКИЙ УНИВЕРСИТЕТ)\\ % название ВУЗа большим шрифтом
    Физтех-школа аэрокосмических технологий}
	
	\vspace{4cm} % отступ 4 см по вертикали
	{\LARGE Отчет о выполнении лабораторной работы 1.1.4}
	
	\vspace{1cm} % ещё 1 см
	{\huge\bf Измерение интенсивнности  \\ радиационного фона}
	
	\vspace{1cm} % ещё 1 см
	\vfill % заполнение по вертикали пустотой (LaTeX сам решит, сколько здесь отступить)
	
    \begin{flushright} % Этот текст будет с правого края страницы
	   {\LARGE Ефремова Татьяна, Б03-503}
    \end{flushright}
    \vfill 
	\today % Дата сборки документа
\end{titlepage}

\section{Аннотация}
Цели работы: применение методов обработки экспериментальных данных для изучения статистических закономерностей при измерении интенсивности радиационного фона

\section{Теоретические сведения}
\subsection{Радиационный фон}
Космические лучи -- это поток частиц, движущихся с высокими энергиям в космическом пространстве. \vspace{0,3cm} \\
Основной величиной, характеризующей количество частиц в космических лучах, является интенсивность $I$. По определению интенсивность есть число частиц, падающих в единицу времени на единичную площадку, перпендикулярную к направлению наблюдения, отнесенное к единице телесного угла (стерадиану). \vspace{0,3cm} \\
Количество падающих частиц в данной работе будет измеряться при помощи счетчика Гейгера-Мюлера (СТС-6). Он представляет собой наполенный газом сосуд с двумя электродами. Частицы космических лучей ионизируют газ, которым наполнен счетчик, а ткже выбивают электроны из его стенок. Образовавшиеся электроны, ускоряясь в сильном электрическом поле между электродами счетчика, соударяются с молекулами газа и выбивают из них новые вторичные электроны. Эти электроны ускоряются электрическим полем и затем ионизицрют молекулы газа. В результате образуется целая лавина электоронов, и через счетчик резко увеличивается ток. Регистрируется частица. \vspace{0,3cm} \\
Поток космических частиц, составляющих значительную часть радиационного фона, изменяется со временем случайным образом. В таком случае характеристиками этой величины являются ее среднее значение и среднеквадратическое отклонение от него.  \vspace{0,3cm} \\
Среднее значения числа частиц, зарегистрированных счетчиком за время $\tau$:
\begin{equation}
\bar{n} = \frac{1}{N} \sum_{i=1}^{N} n_i
\end{equation} \\
Среднеквадратическая ошибка отдельного измерения:
\begin{equation}
\sigma_\text{отд} = \sqrt{\frac{1}{N} \sum_{i=1}^{N} (n_i - \bar{n})^2} \approx \sqrt{\bar{n}},
\end{equation}
где  $N$ -- количество измерений, $n_i$ --  число срабатываний счетчика за $i$-тый отрезок времени $\tau$. \\ \\
Тогда относительная ошибка отдельного измерения:
\begin{equation}
\mathcal{E}_\text{отд} = \frac{\sigma_\text{отд}}{n_i} \approx \frac{1}{\sqrt{n_i}},
\end{equation}
Относительная ошибка в определении среднего $\bar{n}$:
\begin{equation}
\mathcal{E}_{\bar{n}}  = \frac{\sigma_{\bar{n}}}{\bar{n}} = \frac{\sigma_\text{отд}}{\bar{n}\sqrt{N}} \approx \frac{1}{\sqrt{ \bar{n} N}}.
\end{equation}
\subsection{Распределение Пуассона}
Если события происходят с некоторой фиксированной средней интенсивностью, и каждое следующее событие не зависит от предыдущего, то последовательность таких событий называют пуассоновским процессом. Распределение Пуассона описывает вероятность того, что в фиксированном интервале пауссоновского процесса произойдет определенное количество событий. \vspace{0.3cm} \\
Так, вероятность того, что за отрезок времени $\tau$ будет зарегистрировано $n$ частиц:
\begin{equation}
P_n = \frac{n_0^n}{n!}e^{-n_0}
\end{equation}


\section{Используемое оборудование}
Счетчик Гейгера-Мюллера (СТС-6), блок питания, компьютер с интерфейсом связи со счетчиком.

\section{Результаты измерений и обработка данных}

В данном эксперименте будут обработаны данные для 4х времен: $\tau = 10$ с, $\tau = 20$ с, $\tau = 40$ с, $\tau = 80$ с. Сперва проведем наглядно обработку данных для  $\tau = 10$ с.
\begin{table}[h]
\centering
\caption{Число срабатываний счетчика за $\tau = 20$ с}
\begin{tabular}{|c|c|c|c|c|c|c|c|c|c|c|c|c|c|c|c|c|c|c|c|c|}
\hline
№ опыта & 1 & 2 & 3 & 4 & 5 & 6 & 7 & 8 & 9 & 10 \\
\hline
0 & 13 & 24 & 15 & 14 & 27 & 20 & 21 & 15 & 20 & 18 \\
\hline
10 & 24 & 18 & 17 & 15 & 14 & 20 & 20 & 23 & 24 & 40 \\
\hline
20 & 14 & 25 & 19 & 21 & 16 & 26 & 16 & 22 & 24 & 21 \\
\hline
30 & 19 & 15 & 19 & 24 & 26 & 19 & 17 & 21 & 15 & 17 \\
\hline
40 & 22 & 27 & 15 & 22 & 20 & 19 & 24 & 19 & 20 & 24 \\
\hline
50 & 18 & 17 & 10 & 18 & 29 & 29 & 26 & 32 & 28 & 16 \\
\hline
60 & 20 & 27 & 26 & 19 & 13 & 15 & 15 & 24 & 17 & 18 \\
\hline
70 & 20 & 27 & 19 & 30 & 20 & 23 & 20 & 26 & 15 & 14 \\
\hline
80 & 19 & 25 & 27 & 19 & 15 & 21 & 36 & 26 & 14 & 18 \\
\hline
90 & 15 & 21 & 19 & 15 & 20 & 19 & 13 & 23 & 18 & 20 \\
\hline
100 & 23 & 27 & 18 & 16 & 18 & 18 & 17 & 16 & 24 & 17 \\
\hline
110 & 18 & 16 & 25 & 18 & 15 & 17 & 25 & 21 & 13 & 21 \\
\hline
120 & 13 & 23 & 24 & 22 & 9 & 20 & 17 & 27 & 21 & 18 \\
\hline
130 & 20 & 15 & 16 & 23 & 20 & 19 & 19 & 16 & 12 & 17 \\
\hline
140 & 16 & 18 & 21 & 16 & 15 & 30 & 22 & 21 & 29 & 17 \\
\hline
150 & 20 & 15 & 14 & 13 & 22 & 15 & 12 & 18 & 27 & 20 \\
\hline
160 & 15 & 14 & 19 & 10 & 18 & 21 & 22 & 18 & 27 & 16 \\
\hline
170 & 10 & 15 & 19 & 22 & 21 & 19 & 22 & 15 & 21 & 17 \\
\hline
180 & 19 & 14 & 23 & 16 & 15 & 26 & 16 & 16 & 24 & 15 \\
\hline
190 & 19 & 16 & 18 & 19 & 24 & 17 & 21 & 24 & 16 & 33 \\
\hline
\end{tabular}
\end{table}

\begin{table}[h]
\centering
\caption{Данные для построения гистограммы при $\tau = 20$ с}
\begin{tabular}{|c|c|c|c|c|c|c|c|c|c|c|c|c|c|c|c|c|c|c|c|c|c|c|c|}
\hline
$n$  & 3 & 4 & 5 & 6 & 7 & 8 & 9 & 10 & 11 & 12 & 13 \\
\hline
Число случаев & 3 & 1 & 5 & 12 & 11 & 21 & 31 & 33 & 43 & 39 & 43 \\
\hline
Доля случаев $w_n$ & 0.007 & 0.003 & 0.013 & 0.030 & 0.028 & 0.052 & 0.077 & 0.083 & 0.107 & 0.098 & 0.107 \\
\hline
$P_n$ & 0.001 & 0.004 & 0.009 & 0.019 & 0.034 & 0.053 & 0.074 & 0.094 & 0.107 & 0.113 & 0.109 \\
\hline
\end{tabular}
\begin{tabular}{|c|c|c|c|c|c|c|c|c|c|c|c|c|c|c|c|c|c|c|c|c|c|c|c|}
\hline
$n$  & 14 & 15 & 16 & 17 & 18 & 19 & 20 & 21 & 22 & 23 & 24  \\
\hline
Число случаев & 42 & 33 & 20 & 24 & 13 & 9 & 7 & 6 & 1 & 1 & 1 \\
\hline
Доля случаев $w_n$ & 0.105 & 0.083 & 0.050 & 0.060 & 0.033 & 0.022 & 0.018 & 0.015 & 0.003 & 0.003 & 0.003 \\
\hline
$P_n$ & 0.098 & 0.083 & 0.065 & 0.048 & 0.034 & 0.022 & 0.014 & 0.008 & 0.005 & 0.003 & 0.001 \\
\hline
\end{tabular}
\end{table}

\raggedright
Используя полученные данные, можно построить гистограмму вероятностей регистрирования $n$ частиц и сравнить ее с распределением Пуассона для данного отрезка времени.

\centering
\begin{figure}[!h]
   \centering    
    \includegraphics[width=0.95\linewidth]{C:/Users/User/Documents/mipt_takonef/лабы/1.1.4/tau10/гистограмма.png}
    \caption{Гистограмма для $\tau = 10$ с} 
\end{figure}

\raggedright
Аналогично проводится обработка полученных данных для  $\tau = 20$ с, $\tau = 40$ с и $\tau = 80$ с.
\begin{figure}[!h]
   \centering    
    \includegraphics[width=0.95\linewidth]{C:/Users/User/Documents/mipt_takonef/лабы/1.1.4/comparison.png}
    \caption{Сравнение гистограмм для $\tau = 10$ с,  $\tau = 20$ с, $\tau = 40$ с и $\tau = 80$ с} 
\end{figure}

Вычислим среднее число срабатываний счетчика за 10, 20, 40 и 80 секунд. \vspace{0.3cm}\\

\centering
$ \bar{n_{10}} = \frac{1}{400} \sum_{i=1}^{400} n_i \approx 12.6$ \\
$ \bar{n_{20}} = \frac{1}{200} \sum_{i=1}^{200} n_i \approx 25.2$ \\ 
$ \bar{n_{40}} = \frac{1}{100} \sum_{i=1}^{100} n_i \approx 50.4 $ \\
$ \bar{n_{80}} = \frac{1}{50} \sum_{i=1}^{50} n_i \approx 100.8$\\

\raggedright \vspace{0.3cm}
Вычислим среднеквадратическую погрешность измерений для каждого отрезка времени: \\  \vspace{0.3cm}

\centering
$\sigma_{n_{10}} = \sqrt{\frac{1}{400} \sum_{i=1}^{400} (n_i - \bar{n})^2} \approx 3.5$ \\\vspace{0.3cm}
$\sigma_{n_{20}} = \sqrt{\frac{1}{200} \sum_{i=1}^{200} (n_i - \bar{n})^2} \approx 5.0$ \\\vspace{0.3cm}
$\sigma_{n_{40}} = \sqrt{\frac{1}{100} \sum_{i=1}^{100} (n_i - \bar{n})^2} \approx 7.1$ \\ \vspace{0.3cm}
$\sigma_{n_{80}} = \sqrt{\frac{1}{50} \sum_{i=1}^{50} (n_i - \bar{n})^2} \approx 10.0$ \\

\raggedright \vspace{0.3cm}
Вычислим среднеквадратическое отклонение по свойству процесса Пуассона и сравним со стандартной: \\  \vspace{0.3cm}

\centering
$\sigma_{n_{10}} = \sqrt{\bar{n_{10}}} \approx 3.5$ \\\vspace{0.1cm}
$\sigma_{n_{20}} = \sqrt{\bar{n_{20}}} \approx 5.0$ \\\vspace{0.1cm}
$\sigma_{n_{40}} = \sqrt{\bar{n_{40}}} \approx 7.1$ \\\vspace{0.1cm} 
$\sigma_{n_{80}} = \sqrt{\bar{n_{80}}} \approx 10.0$ \\

\raggedright \vspace{0.3cm}
Рассчитаем относительные погрешности: \\  \vspace{0.3cm}

\centering
$ \mathcal{E}_{\bar{n_{10}}} = \frac{100\%}{\sqrt{ \bar{n_{10}} \cdot 400}} \approx 1,4 \%$ \\  \vspace{0.3cm}
$ \mathcal{E}_{\bar{n_{20}}} = \frac{100\%}{\sqrt{ \bar{n_{20}} \cdot 200}} \approx 1,4 \%$ \\ \vspace{0.3cm}
$ \mathcal{E}_{\bar{n_{40}}} = \frac{100\%}{\sqrt{ \bar{n_{40}} \cdot 100}} \approx 1,4 \%$ \\ \vspace{0.3cm}
$ \mathcal{E}_{\bar{n_{80}}} = \frac{100\%}{\sqrt{ \bar{n_{80}} \cdot 50}} \approx 1,4 \%$ \\

\raggedright \vspace{0.3cm}
Окончательный результат:\\  \vspace{0.3cm}

\centering
$n_{\tau=10} = 12,6 \pm 3.5$ \\
$n_{\tau=20} = 25.2 \pm 5.0$ \\
$n_{\tau=40} = 50.4 \pm 7.1$ \\
$n_{\tau=80} = 100.8 \pm 10.0$ \\

\raggedright \vspace{0.3cm}

Для каждого $\tau$ вычислим среднюю интенсивность регистрируемых частиц $\bar{j} = \frac{\bar{n}}{\tau}$ в секунду:\\  \vspace{0.3cm}

\centering
$i_{\tau=10} = 1.26 \pm 0.35$\\
$i_{\tau=20} = 1.26 \pm 0,35$ \\
$i_{\tau=40} = 1.26  \pm 0,35$ \\
$i_{\tau=80} =  1.26\pm 0,35$ \\

\raggedright 
\section{Выводы}
В ходе работы познакомилась с основными понятиями статистики. Определила среднее чмсло регистрируемых космических лучей в секунду и погрешность полученного результата. Выяснила, что средняя интенсивность регистрируемых частиц не завсит от величины интервала $\tau$ и числа точек $N$. 

\end{document}

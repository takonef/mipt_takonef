\documentclass[a4paper]{article}
\usepackage{graphicx} % Required for inserting images
\usepackage{wrapfig}
\usepackage{enumitem}
\usepackage[center]{caption}
\usepackage[english, russian]{babel}
\usepackage{amsmath}
\usepackage{geometry}
\geometry{verbose, a4paper, tmargin=2cm, bmargin=2cm, lmargin=2.0cm, rmargin=2.0cm}

\begin{document}

\begin{titlepage}
	\centering
	{\scshape\Large МОСКОВСКИЙ ФИЗИКО-ТЕХНИЧЕСКИЙ ИНСТИТУТ \\
	(НАЦИОНАЛЬНЫЙ ИССЛЕДОВАТЕЛЬСКИЙ УНИВЕРСИТЕТ)\\ % название ВУЗа большим шрифтом
    Физтех-школа аэрокосмических технологий}
	
	\vspace{4cm} % отступ 4 см по вертикали
	{\LARGE Отчет о выполнении лабораторной работы 1.2.2}
	
	\vspace{1cm} % ещё 1 см
	{\huge\bf Экспериментальная проверка \\ закона вращательного движения \\ \vspace{0.15cm} на крестообразном маятнике}
	
	\vspace{1cm} % ещё 1 см
	\vfill % заполнение по вертикали пустотой (LaTeX сам решит, сколько здесь отступить)
	
    \begin{flushright} % Этот текст будет с правого края страницы
	   {\LARGE Ефремова Татьяна, Б03-503}
    \end{flushright}
    \vfill

	\today % Дата сборки документа
\end{titlepage}

\section{Аннотация}
Цели работы: 1) экспериментально проверить уравнение вращательного движения тела вокруг закрепленной оси, получить зависимость углового ускорения от момента инерции и момента прикладываемых к системе сил; 2) проанализировать влияние сил трения, действующих в оси вращения; 3) определить момент инерции маятника.

\begin{wrapfigure}{R}{0.3\textwidth}
    \includegraphics[width=1\linewidth]{C:/Users/User/Documents/mipt_takonef/лабы/1.2.2/oberbek.PNG}
    \caption{Маятник Обербека}
\end{wrapfigure}

\section{Теоретические сведения}
Уравнение вращательного движения тела вокруг закрепленной оси:
\begin{equation}
I\ddot{\varphi} = M,
\end{equation} где $\ddot{\varphi} = \dot{\omega} = \beta$ -- угловое ускорение ($\omega$ -- угловая скорость), $I$ -- полный
момент инерции тела относительно оси вращения, $M$ -- суммарный момент внешних сил относительной этой оси. \\ \\
Момент силы натяжения нити:
\begin{equation}
M_\text{н} = m_\text{н}r(g-\beta r),    
\end{equation}
где  $m_\text{н} = m_\text{п} + m_\text{п}$ -- масса платформы с перегрузком. \\ \\
Таким образом, с учетом (2) уравнение (1) может быть записано как:
\begin{equation}
(I + m_\text{н}r^2)\beta = m_\text{н}gr - M_\text{тр},
\end{equation} где $M_\text{тр}$ --  момент силы трения в оси вращения. \\ \\
В проведенных опытах $m_\text{н} r^2 << I$,  поэтому можно считать, что маятник раскручивается с постоянным угловым ускорением:
\begin{equation}
    \beta_0 = \frac{1}{I}m_\text{н}gr - \frac{M_{\text{тр}}}{I}
    \label{eq:beta_0}
\end{equation}
Т. к. грузы имеют форму полых цилиндров с внутренним и внешним радиусами и образующей $a_1$ и $a_2$ и  $h$ соответственно, момент инерции системы $I$ вычисляется при помощи теоремы Гюйгенса-Штейнера:
% Requires: \usepackage{amsmath}
\begin{equation}
    I = I_0 + \sum_{i=1}^{4} \left( \frac{1}{12} m_i h^2 + \frac{1}{4} m_i (a_1^2 + a_2^2) + m_i R_i^2 \right).
    \label{eq:moment_of_inertia}
\end{equation}
\section{Используемое оборудование}
Крестообразный маятник, набор перегрузков, секундомер, линейка, весы, штангенциркуль.

Инструментальные погрешности:
\begin{description}
	\item[весы]: $\Delta_\text{в} = 0.1$ г
	\item[штангенциркуль]: $\Delta_\text{шт} = 0.1$ мм
\end{description}


\section{Результаты измерений и обработка данных}
\subsection{Оценка момента силы трения}
\raggedright
Для оценки момента силы трения в подшипниках проверялось наличие движения в системе при отсутствии перегрузков на платформе. Маятник не приходил в движение вплоть до добавления груза массой $6,65$ г. Измерения проводились на шкифе радиусом $R = 17,5$ мм, масса подвеса $m_\text{п} = 6,17$ г. \\
Тогда граничное значение момента силы трения:
$M_0 = (m_\text{гр} + m_\text{гр}) gR \approx 0.0022$ Н $\cdot$ м.
\newpage
\subsection{Измерения углового ускорения}
Измерения углового ускорения прговодились для трех различных положений r грузов на маятнике при 5 различных значениях массы перегрузка. Радиус шкифа $R = 17,5$ мм, масса подвеса $m_\text{п} = 6,17$ г.
\begin{table}[!h]
\centering
\caption{Измерения углового ускорения}
\begin{tabular}{|c|c|c|c|c|c|c|c|}
\hline
\multicolumn{2}{|c|}{} & \multicolumn{2}{|c|}{r = 140 мм} & \multicolumn{2}{|c|}{r = 100 мм} & \multicolumn{2}{|c|}{r = 60 мм}\\
\hline
m, гр & M, Н $\cdot$ м & $\beta$, рад/$c^2$ & $\sigma_\beta$, рад/$c^2$ & $\beta$, рад/$c^2$ & $\sigma_\beta$, рад/$c^2$ &  $\beta$, рад/$c^2$ & $\sigma_\beta$, рад/$c^2$\\
\hline
33,27 & 0,0057 & 0,214 & 0,005 & 0,369 & 0,003 & 0.945 & 0,013 \\
58,17 & 0,0099 & 0,467 & 0,002  & 0,623 & 0,004 & 0.911 & 0,014 \\
106,17 & 0,0182 & 0,609 & 0,002 & 1,227 & 0,003 & 1.784 & 0,020 \\
154,63 & 0,0273 & 1,146 & 0,004 & 1,696 & 0,005 & 2.403 & 0,020 \\
206,17 & 0,0354 & 1,404 & 0,003 & 2,215 & 0,005 & 3.252 & 0,022 \\
\hline
\end{tabular}
\end{table} \\
По результатам измерений можно построить графики зависмостей углового ускрения $\beta$ от момента силы перегрузка $M$:  \\ \centering $\beta= kM-b$, \\ \raggedright где $k = \frac{1}{I}$, $b = \frac{M_\text{тр}}{I}$. \\

Таким образом:
\[
\begin{gathered}
I=\frac{1}{k}, \text{ } \sigma_{I}=I \frac{\sigma_{k}}{k} \\
M_{\text{тр}}=I b, \text{ } \sigma_{{\text{тр}}}=M_{\text{тр}} \sqrt{\left(\frac{\sigma_{b}}{b}\right)^{2}+\left(\frac{\sigma_{I}}{I}\right)^{2}}
\end{gathered}
\]

Полученные значения представлены в таблице 2, полученные зависимости изображены на рисунке 2.  \\
\begin{table}[!h]
    \centering
    \begin{tabular}{|c|c|c|c|c|c|c|c|c|c|c|}
    \hline
    $R$, мм & $k$, $\frac{1}{\text{кг}\cdot \text{м}^2}$ & $\sigma_k$, $\frac{1}{\text{кг}\cdot \text{м}^2}$ & $b$, $\frac{\text{Н}}{\text{кг} \cdot \text{м}}$ & $\sigma_b$, $\frac{\text{Н}}{\text{кг} \cdot \text{м}}$ & $I$, $\text{кг} \cdot \text{м}^2$ & $\sigma_I$ & $M_\text{тр}$, $\text{H} \cdot \text{м}$ & $\sigma_{\text{тр}}$ \\
    \hline
    60 & 80.8 & 2.53 &  0.298 & 0.135 & 0.0124 & 0.0005 & 0.0035 & 0.0015 \\
    100 & 62.0 & 1.43 & 0.137 & 0.054 & 0.0171 & 0.0003 & 0.0024 & 0.0009 \\
    140 & 39.9 & 0.99 & 0.093 & 0.035 & 0.0250 & 0.0006 & 0.0022 & 0.0008 \\
    \hline
    \end{tabular}
    \caption{Значения коэфицентов прямых, моментов инерции и моментов силы трения}
\end{table}
$\bar{M_\text{тр}} = 0.0027 \pm 0.0010$ $\text{H} \cdot \text{м}$.
Получается, измеренное $M_0 = 0.0022 \text{ H} \cdot \text{м}$ лежит в пределах погрешности.

\begin{figure}[!h]
    \includegraphics[width=1\linewidth]{C:/Users/User/Documents/mipt_takonef/лабы/1.2.2/plots_oberbek.PNG}
    \caption{Зависимости углового ускорения от момента сил тяжести}
\end{figure}
\subsection{Вычисление момента инерции маятника}
$m_1 = 151.7$ г,
$m_2  =147.6$ г,
$m_3 = 147.7$ г,
$m_4  = 149.1$ г,
$a_1 = 3.8$ мм,
$a_2 = 17.5$ мм,
$h = 25$ мм.

\begin{equation}
    I_0 = I - \sum_{i=1}^{4} \left( \frac{1}{12}m_i h^2 + \frac{1}{4}m_i (a_1^2 + a_2^2) + m_i R_i^2 \right)
    \label{eq:placeholder}
\end{equation}

Выражение имеет вид $I = b + kR^2$. Тогда по МНК можно вычислить коэффициенты b и k:
$k = 1.26 \pm 0.02$,  $b = -0.0118 \pm 0.0002 $ \\

Поскольку массы грузов и расстояния почти не отличаются, $ I_i = \frac{1}{12}m_i h^2 + \frac{1}{4}m_i (a_1^2 + a_2^2) \approx 4I_1$
$4I_1 \approx 79 \cdot 10^{-6}$, что на порядок меньше коэфиициента $b$. Тогда $|b| \approx I_0$, $\sigma_b \approx \sigma_{I_0}$ \\ \vspace{0.2cm}

Рассчитаем значения момента инерции маятника по формуле (6): \\
$I_0(60 \text{мм}) = 0.0097$ $\text{кг} \cdot \text{м}^2$,
$I_0(100 \text{мм}) = 0.0109$ $\text{кг} \cdot \text{м}^2$,
$I_0 (140 \text{мм}) = 0.0122$ $\text{кг} \cdot \text{м}^2$. \\

Тогда: \\
$\bar{I_0} = 0,0109 \pm 0.0002$ $\text{кг} \cdot \text{м}^2$

\section{Выводы}
Полученные в ходе работы значения собственного момента инерции маятника для разных положений грузов приблизительно равны. Для разных моментов инерции маятника вычисленные моменты силы трения в оси оказались приблизительно равными, что соответствует действительности и подтверждает справедливость используемых формул и допустимых приближений. Точности эксперимента достаточно, чтобы проверить все рассмотренные в работе теоретические закономерности, однако для измерения моментов инерции предпочтительнее другие методы, так как ошибка в данном опыте слишком велика для более точных измерений.

\end{document}

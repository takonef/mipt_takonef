\documentclass[a4paper]{article}
\usepackage{graphicx} % Required for inserting images
\usepackage{wrapfig}
\usepackage{enumitem}
\usepackage[center]{caption}
\usepackage[english, russian]{babel}
\usepackage{amsmath}
\usepackage{geometry}
\usepackage{multirow}
\geometry{verbose, a4paper, tmargin=2cm, bmargin=2cm, lmargin=2.0cm, rmargin=2.0cm}

\begin{document}

\begin{titlepage}
	\centering
	{\scshape\Large МОСКОВСКИЙ ФИЗИКО-ТЕХНИЧЕСКИЙ ИНСТИТУТ \\
	(НАЦИОНАЛЬНЫЙ ИССЛЕДОВАТЕЛЬСКИЙ УНИВЕРСИТЕТ)\\ % название ВУЗа большим шрифтом
    Физтех-школа аэрокосмических технологий}
	
	\vspace{4cm} % отступ 4 см по вертикали
	{\LARGE Отчет о выполнении лабораторной работы 1.2.5}
	
	\vspace{1cm} % ещё 1 см
	{\huge\bf Измерение вынужденной регулярной \\прецессии гироскопа}
	
	\vspace{1cm} % ещё 1 см
	\vfill % заполнение по вертикали пустотой (LaTeX сам решит, сколько здесь отступить)
	
    \begin{flushright} % Этот текст будет с правого края страницы
	   {\LARGE Ефремова Татьяна, Б03-503}
    \end{flushright}
    \vfill

	\today % Дата сборки документа
\end{titlepage}

\section{Аннотация}
Цели работы: исследовать вынужденную прецессию гироскопа; установить зависимость скорости вынужденной прецессии от величины момента сил, действующих на ось гироскопа; определить скорость вращения ротора гироскопа и сравнить её со скоростью, рассчитанной по скорости прецессии.

\section{Теоретические сведения}

Уравнения движения твердого тела можно представить в виде:
\begin{equation}
\frac{d\vec{P}}{dt} = \vec{F},
\end{equation}
\begin{equation}
\frac{d\vec{L}}{dt} = \vec{M}.
\end{equation} \\
Здесь (1) выражает закон движения центра масс тела, а (2) – уравнение моментов. Если сила $\vec{F}$ не зависит от угловой скорости, а момент сил $\vec{M}$ -- от скорости поступательного движения, то уравнения можно рассматривать независимо друг от друга. В данной работе для описания движения гироскопа потребуется только уравнение (2). \vspace{0.25 cm} \\
Момент импульса твёрдого тела в его главных осях можно выразить по формуле
\begin{equation}
\vec{L} = \vec{i}I_x\omega_x + \vec{y}I_y\omega_y + \vec{k}I_z\omega_z,
\end{equation} 
где $I_x, I_y, I_z$ – главные моменты инерции, а $\omega_x, \omega_y, \omega_z$ – компоненты вектора угловой скорости $\vec{\omega}$. \vspace{0.25 cm} \\ 
Быстро вращающееся тело, для которого произведение момента инерции на компоненту угловой скорости одной из осей много больше произведений для двух других осей, принято называть гироскопом.  \vspace{0.25 cm} \\
В силу (2) приращение момента импульса определяется интегралом
\begin{equation} 
\Delta \vec{L} = \int\vec{M} dt.
\end{equation}
Если момент внешних сил действует в течение короткого промежутка времени, из интеграла (4) следует, что приращение $\Delta\vec{L}$ момента импульса значительно меньще самого момента импульса. С этим связана устойчивость гироскопа, привденного в быстрое вращение. \vspace{0.25 cm}

\begin{wrapfigure}{R}{0.3\textwidth}
    \includegraphics[width=1\linewidth]{C:/Users/User/Documents/mipt_takonef/лабы/1.2.5/1.2.5_scheme.PNG}
    \caption{Маховик}
\end{wrapfigure}
\noindent Рассмотрим маховик, вращающийся вокруг оси $z$, перпендикулярной к его плоскости. Будем считать, что
\[
\begin{gathered}
\omega_z = \omega_0, \text{ } \omega_y = 0, \text { } \omega_x = 0.
\end{gathered}
\]
Пусть ось вращения повернулась в плоскости $zx$ по направлению к оси $x$ на бесконечно малый угол $d\varphi$. Такой поворот означает добавочное вращение маховика вокруг оси $y$, так что 
\[
\begin{gathered}
d\varphi = \Omega dt,
\end{gathered}
\]
где $\Omega$ -- угловая скорость вращения. Если $L_\Omega$ много меньше $L_{\omega_0}$, то момент импульса маховика лишь повернется в плоскости $zx$ по направлению к оси $x$, не изменяя своей величины. Тогда изменение направлено вдоль оси $x$, и вектор $\vec{L}$ можно представить в виде векторного произведения вектора угловой скорости $\vec{\Omega}$ на вектор собственного момента испульса маховика $\vec{L}$. Таким образом, 
\begin{equation}
\frac{d\vec{L}}{dt} = \vec{\Omega} \times \vec{L}.
\end{equation}
Окончательно, в силу (2)
\begin{equation}
\vec{M} = \vec{\Omega} \times \vec{L}.
\end{equation} \\

\newpage
\noindent Под действием момента $\vec{M}$ внешних сил ось гироскопа медленно вращается вокруг оси $y$ с угловой скоростью $\Omega$. Такое движение называется регулярной прецессией гироскопа. В данной работе для ее изучения на ось вращения гироскопа будут подвешены дополнительные грузы. Прецессию в таком случае вызывает момент силы тяжести, а скорость ее равна
\begin{equation}
\Omega = \frac{mgl}{I_z\omega_0},
\end{equation}
где $m$ -- масса груза, $l$ -- расстояние от центра карданового подвеса до точки крепления груза. \vspace{0.25 cm} \\
Вычисление момента инерции ротора $I_z$ относительно оси вращения гироскопа происходит при помощи крутильного маятника. Так как период крутильных колебаний $T_0$ зависит лишь от момента инерции тела $I_0$ и модуля кручения проволоки
\[
\begin{gathered}
T_0 = 2\pi \sqrt{\frac{I_0}{f}},
\end{gathered}
\]
момент инерции $I_z$ можно рассчитать как
\begin{equation}
I_z = I_\text{ц} \frac{T_z^2}{T_\text{ц}^2},
\end{equation}
где $I_\text{ц}$ -- момент инерции циллиндра известных массы и диаметра; $T_\text{ц}, T_z$ -- периоды крутильных колебаний.

\section{Оборудование}
\subsection{Используемое оборудование}
Гироскоп в кардановом подвесе, секундомер, набор грузов, отдельный ротор гироскопа, цилиндр известной массы, крутильный маятник, штангенциркуль.

\subsection{Инструментальные погрешности}
\vspace{0.15 cm}

\setdescription{leftmargin=0pt}
\begin{description}
    \item[весы]: $\Delta_\text{в} = \pm 0.1$ г.
    \item[штангенциркуль:] $\Delta_\text{шт}=\pm0,1$ мм.
    \item[секундомер:] $\Delta_\text{с}=\pm0,1$ с.
\end{description}

\section{Результаты измерений и обработка данных}

\subsection{Скорость вынужденной прецессии гироскопа}
В данной работе скорость прецессии измеряется через период полного оборота гироскопа вокруг вертикальной оси при воздействии на него момент силы тяжести. Измерения проводятся для пяти грузов различного веса.  Из среднего значения периода оборота $\overline{T}$ вычисляется угловая скорость прецессии.
\[
\begin{gathered}
\overline{T} = \frac{1}{N}  \sum_{i=1}^N T_i ; \text{ }\Omega = \frac{2\pi}{\overline{T}}
\end{gathered}
\]
\[
\begin{gathered}
\sigma_{\overline{T}} = \sqrt{ {\frac{1}{(N-1)^2}} \sum_{i=1}^N (T_i - \overline{T})^2 + \Delta_\text{с}^2}; \text{ } \sigma_{\Omega} = 2\pi \Omega \frac{\sigma_{\overline{T}}}{\overline{T}}
\end{gathered}
\]

\begin{table}[h]
\caption{Число и время оборотов} 
\centering

\begin{tabular}{|c|c|c|c|c|c|c|c|c|c|c|}
\hline
$m$, г & $M$, Н $\cdot$ м & $T_1,$ c & $T_2,$ c & $T_3,$ c & $T_4,$ c & $T_5,$ c & $\overline{T},$ с & $\sigma_{\overline{T}},$ с & $\Omega,$ рад/с & $\sigma_{\Omega},$ рад/с \\
\hline
56,0 & 664 & 183,5 & 184,2 & 184,8 & 181,0 & 182,9 & 183,9 & 0,8 & 0,034 & 0,001\\
\hline
91,4 & 1084 & 109,4 & 110,3 & 111,0 & 110,5 & 110,1 & 110,3 & 0,3 & 0,057 & 0,001\\
\hline
178,9 & 2123 & 56,7 & 56,0 & 56,2 & 56,0 & 56,1  & 56,2 & 0,2 & 0,112 & 0,002\\
\hline
272,1 & 3229 & 36,9 & 36,8 & 36,8 & 37 & 36,7 & 36,8 & 0,1 & 0,171 & 0,003\\
\hline
340,4 & 4040 & 29,6 & 29,1 & 29,4 & 29,5 & 29,4 & 29,4 & 0,1 & 0,214 & 0,006\\
\hline
\end{tabular}
\end{table}

\noindent Т. к. зависимость $\Omega$ от $M = mgl$, где $l = 1,21$ м, линейна, по методу наименьших квадратов,
\[
\begin{gathered}
k = \frac{1}{I_z \omega_0} = \frac{\langle M\Omega\rangle}{\langle M^2 \rangle}, \text{ } \sigma_k = \frac{1}{\sqrt{N-1}}\sqrt{\frac{ \langle \Omega^2 \rangle }{ \langle M \rangle } - k^2}.
\end{gathered}
\]
Тогда $\frac{1}{I_z\omega_0} = 0,531 \pm 0,005$ с/кг.
\begin{figure}[!h]
   \centering    
    \includegraphics[width=1\linewidth]{C:/Users/User/Documents/mipt_takonef/лабы/1.2.5/125_plot.png}
    \caption{Линейная аппроксимация зависимости скорости прецессии \\ от момента силы тяжести методом наименьших квадратов} 
\end{figure}

\subsection{Момент инерции ротора}

Период крутильных колебаний ротора на проволоке составил $T_0 = 3,2$ с. Для цилиндра $T_\text{ц} = 4,0$ с. Радиус цилиндра $R_\text{ц} = 39,0$ мм, масса -- $m_\text{ц} = 1615,4$ г. Тогда его момент инерции:
\[
\begin{gathered}
I_\text{ц} = \frac{m_\text{ц} r^2}{2} \approx 1,232 \cdot 10^{-3} \text{ кг} \cdot \text{м}^2.
\end{gathered}
\]
Из соотношения (8) момент инерции ротора:
\[
\begin{gathered}
I_z = I_\text{ц} \frac{T_0^2}{T_\text{ц}^2} \approx 0,791  \cdot 10^{-3} \text{ кг} \cdot \text{м}^2.
\end{gathered}
\]
Погрешности вычисления моментов инерции:
\[
\begin{gathered}
\sigma_{I_{\text{ц}}} = \frac{1}{2} I_{\text{ц}} \sqrt{2(\frac{\Delta_{\text{шт}}}{R})^2 + (\frac{\Delta_{\text{в}}}{m_{\text{ц}}})^2} \approx 0,001 \cdot 10^{-3}  \text{ кг} \cdot \text{м}^2,
\end{gathered}
\]
\[
\begin{gathered}
\sigma_{I_{z}} = I_{z} \sqrt{2(\frac{\Delta_{c}}{T_{\mathrm{ц}}})^{2} + 2(\frac{\Delta_{c}}{T_{0}})^{2} + (\frac{\sigma_{I_{\mathrm{ц}}}}{I_{\mathrm{ц}}})^{2}} \approx 0,003 \cdot 10^{-3}  \text{ кг} \cdot \text{м}^2.
\end{gathered}
\]

\subsection{Угловая скорость вращения ротора гироскопа}

\[
\begin{gathered}
\omega_0 = \frac{1}{kI_z} \approx 2383 \text{ рад/c}; \text{ } \sigma_{\omega_{0}} = \omega_{0} \sqrt{(\frac{\sigma_{k}}{k})^{2} + (\frac{\sigma_{I_{z}}}{I_{z}})^{2}} = 9 \text{ рад/с}.
\end{gathered}
\]
Частота вращения ротора гироскопа также измерялась осциллографом при помощи фигур Лиссажу: $\nu \approx 380$ Гц. Отсюда $\omega = 2 \pi \nu \approx 2387$ рад/с.
\section{Выводы}
Полученные разными методами значения угловой скорости вращения ротора совпали в пределах погрешности. При этом относительная погрешность полученного результата невелика:  $\varepsilon \approx 0,4 $\%. Все теоретические закономерности, рассмотренные в данной работе, также выполнились в пределах точности эксперимента.
\end{document}

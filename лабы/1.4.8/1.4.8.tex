\documentclass[a4paper]{article}
\usepackage{graphicx} % Required for inserting images
\usepackage{wrapfig}
\usepackage{enumitem}
\usepackage[center]{caption}
\usepackage[english, russian]{babel}
\usepackage{amsmath}
\usepackage{geometry}
\geometry{verbose, a4paper, tmargin=2cm, bmargin=2cm, lmargin=2.0cm, rmargin=2.0cm}

\begin{document}

\begin{titlepage}
	\centering
	{\scshape\Large МОСКОВСКИЙ ФИЗИКО-ТЕХНИЧЕСКИЙ ИНСТИТУТ \\
	(НАЦИОНАЛЬНЫЙ ИССЛЕДОВАТЕЛЬСКИЙ УНИВЕРСИТЕТ)\\ % название ВУЗа большим шрифтом
    Физтех-школа аэрокосмических технологий}
	
	\vspace{4cm} % отступ 4 см по вертикали
	{\LARGE Отчет о выполнении лабораторной работы 1.4.8}
	
	\vspace{1cm} % ещё 1 см
	{\huge\bf Определение модуля Юнга методом аккустического резонанса}
	
	\vspace{1cm} % ещё 1 см
	\vfill % заполнение по вертикали пустотой (LaTeX сам решит, сколько здесь отступить)
	
    \begin{flushright} % Этот текст будет с правого края страницы
	   {\LARGE Ефремова Татьяна, Б03-503}
    \end{flushright}
    \vfill

	\today % Дата сборки документа
\end{titlepage}

\section{Аннотация}
Цели работы: исследовать явление акустического резонанса в тонком стержне; измерить скорость распространения продольных звуковых колебаний в тонких стержнях из различных материалов и различных размеров; измерить модули Юнга различных материалов.

\section{Теоретические сведения}

Основной характеристикой упругих свойств твёрдого тела является его модуль Юнга $E$. Согласно закону Гука, если к элементу среды приложено некоторое механическое напряжение $\sigma$, действующее вдоль некоторой оси $x$ (напряжения по другим осям при этом отсутствуют), то в этом элементе возникнет относительная деформация вдоль этой же оси $E = \Delta x/x_0$. \vspace{0.2 cm} \\
Если с помощью кратковременного воздействия в некотором элементе твёрдого тела создать малую деформацию, она будет далее распространяться в среде в форме акустической волны. Волны, распространяющиеся вдоль оси, по которой происходит деформация, называются продольными. Скорость $u$ распространения такой волны в простейшем случае длинного тонкого стержня определяется соотношением
\begin{equation}
u = \sqrt{\frac{E}{\rho}},
\end{equation}
где $\rho$ -- плотность среды.  \vspace{0.2 cm} \\
В общем случае звуковые волны в твёрдых телах могут быть не только продольными, но и поперечными (деформация сдвига перпендикулярна распространению волны), однако в данной работе будет исследован наиболее простой случай упругих волн, распространяющихся в длинных тонких стержнях. Если длины волны $\lambda$ и стержня $L$ много больше его радиуса $R$, то такая волна может свободно распространяться лишь вдоль стержня, и его упругие свойства описываются только модулем Юнга. \vspace{0.2 cm} \\ 
Акустическая волна, распространяющаяся в стержне конечной длины $L$, отражается от торцов стержня. Если при этом на длине стержня укладывается целое число полуволн, то отражённые волны будут складываться в фазе с падающими, что приведёт к резкому усилению амплитуды их колебаний и возникновению акустического резонанса в стержне. Измеряя соответствующие резонансные частоты, можно определить скорость звуковой волны в стержне и, таким образом, измерить модуль Юнга его материала. \vspace{-0.2 cm} \\
Скорость u определяется как:
\begin{equation}
u = 2L\frac{f_n}{n},
\end{equation} 
где $n$ -- номер гармоники. \vspace{0.2 cm} \\
Согласно теории, зависимость $f_n(n)$ линейна, и для всех резонансных частот отношение $\frac{f_n}{n}$ постоянно. Однако, если в идеальном случае резонанс достигался бы при строгом совпадении частот $f = f_n$, в реальности возбуждение стоячей волны возможно при относительно малом отклонении частоты от резонансной. Амплитуда как функция частоты $A(f)$ имеет резкий максимум при $f = f_n$. При этом ширина резонансного максимума $\Delta f$ определяется добротностью $Q$ колебательной системы:
\begin{equation}
\Delta f \approx \frac{f_\text{рез}}{Q}
\end{equation}
Именно конечная ширина резонанса $\Delta f$ определяет в основном погрешность измерения частоты.

\section{Оборудование}
\subsection{Используемое оборудование}
Генератор звуковых частот, частотомер, осциллограф, электромагнитные излучатель и приёмник колебаний, набор стержней из различных материалов, малые циллиндры из различных материалов, весы, штангенциркуль, микрометр.
\subsection{Инструментальные погрешности}
\vspace{0.15 cm}

\setdescription{leftmargin=0pt}
\begin{description}
    \item[весы]: $\Delta_\text{в} = \pm 0.001$ г.
    \item[штангенциркуль:] $\Delta_\text{шт}=\pm0,1$ мм.
    \item[микрометр:] $\Delta_\text{мкм}=\pm0,01$ мм.
\end{description}

\section{Результаты измерений и обработка данных}

\subsection{Добротность стержней}
Ширина максимума функции $A(f-f_n)$ связана с добротностью $Q$ стержня как колебательной системы: если $\Delta f$ — ширина амплитудно-частотной характеристики на уровне $A = \frac{A_{max}}{\sqrt2}$, то $Q = \frac{f_n}{\Delta f}$. 
\begin{table}[h]
\caption{Частоты максимума амплитуды; добротность} 
\centering
\begin{tabular}{|c|c|c|c|c|c|c|c|c|c|c|}
\hline
& $f(A_{max})$, Гц & $f_1(0,7A_{max}) $, Гц & $f_2(0,7A_{max})$ , Гц & $\Delta f$, Гц & Q\\
\hline
Медь & 3218.6 & 3218.0 & 3219.2 & 1.2 & 2682\\
\hline
Сталь & 4131.1 & 4130.2 & 4132.0 & 1.8 & 2295\\
\hline
Дюраль & 4254.0 & 4253.0 & 4255.5 & 2.5 & 1701\\
\hline
\end{tabular}
\end{table} \\
Используемые в работе металлические стержни являются весьма высокодобротными системами; ширина резонанса мала, то есть и погрешность поиска резонансных частот -- тоже. Тем не менее, время установления резонансных колебаний, которое можно оценить как $\tau_\text{уст} \approx \frac{1}{\Delta f} $, оказывается достаточно велико (до нескольких секунд), из-за чего поиск резонанса следует проводить, изменяя частоту генератора максимально медленно.

\subsection{Измерение скорости $u$ распространения аккустрических волн}
Так как в реальном стержне могут возбуждаться как продольные, так и поперечные колебания, сопровождающиеся множеством «паразитных» частот, для выделения нужных резонансов проводится предварительный анализ. \vspace{0.2 cm} \\
Первую резонансную частоту можно оценить по формуле $f_\text{теор} = \frac{U}{2L}$, где $U \approx 3.7 \cdot 10^3$ м/с. \\
Тогда $f_\text{теор} \approx 3083.3$ Гц. Последующие резонансные частоты можно оценить как $f_n \approx n \cdot f_1$.
\begin{table}[h]
\caption{Частоты резонанса для стержней различных материалов} 
\centering
\begin{tabular}{|c|c|c|c|c|c|c|c|c|c|c|}
\hline
& 1 & 2 & 3 & 4 & 5 & 6 & 7 \\
\hline
$f_\text{меди},$ кГц & 3.219 & 6.083 & 9.666 & 12.880 & 16.104 & 19.228 & 21.877 \\
\hline
$f_\text{стали},$ кГц & 4.131 & 8.276 & 12.411 & 16.529 & 20.681 & 25.355 & 28.196 \\
\hline
$f_\text{дюрали},$ кГц & 4.254 & 8.497 & 12.766 & 17.026 & 21.535 & 24.989 & 29.101 \\
\hline
\end{tabular}
\end{table}
\begin{figure}[!h]
   \centering    
    \includegraphics[width=0.95\linewidth]{C:/Users/User/Documents/mipt_takonef/лабы/1.4.8/1.4.8_plot.png}
    \caption{Линейная аппроксимация результатов измерения частот резонанса f \\ в зависимости от значения n методом наименьших квадратов} 
\end{figure} \\
Т. к. зависимость $f_n$ от $n$ линейная, по методу наименьших квадратов:
\[
\begin{gathered}
k = \bar{\left(\frac{f_n}{n}\right)} = \frac{\langle nf_n\rangle}{\langle n^2 \rangle}, \text{ } \sigma_k = \frac{1}{\sqrt{N}}\sqrt{\frac{ \langle f_n^2 \rangle }{ \langle n \rangle } - k^2}
\end{gathered}
\]
Тогда
\[
\begin{gathered}
u = 2Lk, \text{ } \sigma_u=u \sqrt{\left(\frac{\sigma_{L}}{L}\right)^{2}+\left(\frac{\sigma_k}{k}\right)^{2}}
\end{gathered}
\]

\begin{table}[!h]
\caption{Скорости распространения волн} 
\centering
\begin{tabular}{|c|c|c|c|c|c|c|c|c|c|c|}
\hline
& $k $, Гц & $\sigma_k$, Гц & $u$, м/с & $\sigma_u$, м/с \\
\hline
Медь & 3168 & 41 & 3801 & 49 \\
\hline
Сталь & 4094 & 64 & 4913 & 77 \\
\hline
Дюраль & 4153 & 45 & 4983 & 54\\
\hline
\end{tabular}
\end{table} 

\subsection{Измерение плотности стержней}

\[
\begin{gathered}
\rho = \frac{m}{\pi R^2 \cdot L}; \text{ } \sigma_{\rho} = \rho \sqrt{\frac{\Delta_\text{в}}{m}^2 + 2\frac{\Delta_\text{мк}}{R}^2 + \frac{\Delta_\text{шт}}{L}^2}
\end{gathered}
\]

\begin{table}[!h]
\caption{Характеристики различных материалов} 
\centering
\begin{tabular}{|c|c|c|c|c|c|c|c|c|c|c|}
\hline
& L, мм & R, мм & m, г & $\rho$, кг/м$^3$ & $\sigma_{\rho}$, кг/м$^2$\\
\hline
Медь & 40.0 & 6.04 & 40.955 & 8904 & 30\\
\hline
Сталь & 41.0 & 6.15 & 36.902 & 7562 & 25\\
\hline
Дюраль & 41.3 & 6.05 & 13.224 & 2780 & 9\\
\hline
\end{tabular}
\end{table} 

\subsection{Вычисление модуля Юнга}
Из формулы (1):
\[
\begin{gathered}
E = \rho \cdot u^2; \text{ } \sigma_E = \sqrt{\frac{\sigma_{\rho}}{\rho} + 2\frac{\sigma_u}{u}}
\end{gathered}
\]
$E_\text{меди} = 128.6 \pm 2.4$ ГПа; $E_\text{стали} = 182.5 \pm 4.1$ ГПа; $E_\text{дюрали} = 69.0 \pm 1.1$ ГПа.

\section{Выводы}
В результате работы были измерены Модули Юнга для меди, стали и дюрали с точностью не меньше $2,2 \%$. Все теоретические закономерности, рассмотренные в данной работе, также выполнились в пределах точности эксперимента.

\end{document}
